\documentclass{article}
\usepackage{hyperref}
\newcommand{\Spheral}{{\tt Spheral++}}
\newcommand{\curl}{{\tt curl}}
%\newcommand{\url}[1]{{\tt html://#1}}

%-------------------------------------------------------------------------------
\begin{document}
\title{Configuring, Building, and Testing \Spheral}
\author{J. Michael Owen}
\maketitle

%-------------------------------------------------------------------------------
\section{Quickstart for the impatient}
\label{quick.sec}
On many systems you can build \Spheral\ with only a few steps.

To build \Spheral\ for use on a parallel cluster (assuming MPI is available),
one could use:
\begin{verbatim}
cd spheral/src
./boot
./configure
make
\end{verbatim}

For a build to run on a serial computer, you need to explicitly tell
\Spheral\ not build with MPI:
\begin{verbatim}
cd spheral/src
./boot
./configure --without-mpi
make
\end{verbatim}
Notes:
\begin{itemize}
\item \Spheral's build system assumes GNU make.
\item If you are building on a multi-core (or multi-processor) node, you can
  accelerate the build process by passing the available number of cores or
  processors to make with the \verb+-j+ option, i.e. \verb+make -j 2+ on a 2
  core CPU for instance.
\item \Spheral\ relies upon several third party software libraries, listed in \S
  \ref{thirdParty.sec}.  The current version of \Spheral\ downloads and installs
  most of those libraries itself as part of the build process using the program
  \curl.  This makes the initial build of \Spheral\ more expensive and requires
  that you be connected to the Internet so these packages can be fetched and
  built.  Fortunately these library sources are not updated all that often, so
  generally subsequent builds of the source do not have to repeat this step.
  You also must ensure that \curl\ is available, and if not install it (see
  \S\ref{thirdParty.sec}).
\end{itemize}

%-------------------------------------------------------------------------------
\section{Detailed build instructions.}

%-------------------------------------------------------------------------------
\subsection{Generic configuration.}
\Spheral\ uses a fairly standard autoconf based build system originally
provided by Martin Casado.  Executing this system consists of running the
following commands in the {\tt spheral/src} directory:
\begin{verbatim}
./boot
./configure <options>
\end{verbatim}
{\tt configure} has many options, which can be listed with the command
\begin{verbatim}
./configure --help
\end{verbatim}
The options you will most likely need to set are whether to build the MPI
parallel enabled version of \Spheral, the compiler choices, optimization level,
and whether or not you want the Design By Contract (DBC) correctness testing
enabled.

\subsubsection{Selecting whether to build a parallel or a serial code.}
\Spheral\ defaults to building the parallel (MPI-enabled) code.  If you want to
build the serial version only, then specify the
\begin{itemize}
\item\verb+--without-mpi+
\end{itemize}
option.

\subsubsection{Selecting the compiler(s)}
\Spheral\ depends on several compilers for different components, so we have
tried to simplify the compiler selection process through the use of the
\begin{itemize}
\item \verb+--with-compilers=<gnu,intel,vacpp,emsolve>+
\end{itemize}
{\tt configure} option.  Currently the only valid options are {\tt gnu}, {\tt
  intel}, {\tt vacpp}, or {\tt emsolve} -- the default is {\tt gnu}.  The {\tt
  intel} option selects the {\tt icc} compiler suite, only appropriate on intel
environments where this compiler has been installed.  Similarly {\tt vacpp} runs
the {\tt xlC} compile suite on IBM AIX systems.  At the time of this writing,
however, {\tt xlC} does not correctly build the {\tt Boost} third party library
set (\S \ref{free.sec}.\ref{boost.item}) , so this option cannot be used.  The
{\tt emsolve} compiler suite is a very specialized option that will probably not
be of general use.

If the set of compilers selected by one of the above options is not quite right
for a particular installation, it is also possible to specify the individual
compiler choices via the following set of {\tt configure} flags.
\begin{itemize}
\item \verb+--with-CC=<compiler>+: Select the serial C compiler.
\item \verb+--with-CXX=<compiler>+: Select the serial C++ compiler.
\item \verb+--with-MPICC=<compiler>+: Select the MPI CC compiler, only necessary
if you're building the parallel version of \Spheral.
\item \verb+--with-MPICXX=<compiler>+: Select the MPI C++ compiler, also only
necessary if you're building the parallel code.
\item \verb+--with-python-CC=<compiler>+: \ref{free.sec}.\ref{python.item}
 Manually select the C compiler to build python with.  Defaults to the
 C compiler selected by \verb+--with-CC+.
\item \verb+--with-python-CXX=<compiler>+: \ref{free.sec}.\ref{python.item}
Manually select the C++ compiler to build python with.  Defaults to
the C++ compiler selected by \verb+--with-CXX+.
\end{itemize}

\subsubsection{Options affecting optimization level}
Options that are useful for determining the optimization are
\begin{itemize}
\item \verb+--with-opt=<0,1,2,3,4,5,6>+: Select the optimization level.  Level 0
results in an unoptimized debugable executable, which will run quite slowly.
Progressively higher numbers select progressively more aggressive optimization,
which may not work with all compiler choices.  We have experimented with
optimizations up to the maximum with the Gnu compilers, but in general pushing
much beyond 3 does not result in noticeable run time improvements.  Running with
opt set to 2 or 3 generally results in code that is 5-10x faster than level 0.
The default for most compilers is a reasonable level of optimization, but not
necessarily the highest possible value.
\item \verb+--with-dbc=<all,pre,none>+: Select the level at which to enforce
Design By Contract correctness testing.  Possible values are
\begin{itemize}
\item \verb+all+: enforce all contracts
\item \verb+pre+: enforce only preconditions
\item \verb+none+: turn off all contract testing
\end{itemize}
The default value is \verb+none+, which is appropriate for building an
executable optimized for run-time speed.  If you are encountering problems (or
developing new code) however, it is a good idea to enable the contracts with
\verb+all+, which gives exhaustive correctness checking during execution (at the
price of running $\sim 5\times$ more slowly).
\end{itemize}

\subsubsection{ANEOS}
If you have access to the ANEOS (Analytic Equation Of State) package, you can
optionally build \Spheral\ bindings to use that equation of state.  In order to
do this you must specify two options
\begin{itemize}
\item \verb+--with-aneos+: Build the Spheral interface to the ANEOS library.
\item \verb+--with-aneos-link=ARG+: Specify how to link with the ANEOS library.
\end{itemize}
An example configuration to build an optimized \Spheral\ that can use ANEOS
might look like
\begin{verbatim}
./configure --with-opt=3 --with-dbc=none --with-aneos \
            --with-aneos-link="-L/usr/local/ANEOS -lmaneos"
\end{verbatim}
assuming you have a library called \verb+libmaneos.a+ in directory
\verb+/usr/local/ANEOS+.

Note, using ANEOS requires a suitable fortran compiler be available!  ANEOS is a
fortran package, and Spheral builds a few fortran interface files to call into
it if employing this option.

\subsubsection{Miscellaneous options}
Some lesser used options that may be of interest include
\begin{itemize}
\item \verb+--with-debug+: Turn on debug print statements.  Beware, this is a
very verbose option!
\item \verb+--without-triangle+: Disable building the Triangle extensions to the
  polytope tessellation package.
\item \verb+--without-tetgen+: Disable building the Tetgen extensions to the
  polytope tessellation package.
\end{itemize}

%-------------------------------------------------------------------------------
\subsection{Mac OS X.}
\Spheral\ will build and run just fine on the Apple Mac architecture using
Apple's \verb+Xcode+ compiler suite.  The following steps have proven
successful.
\begin{enumerate}
\item Install Apple's \verb+Xcode+ (available in the App Store), which provides
  the necessary compiler suite.  Unfortunately the default Apple version of
  \verb.g++. has some RTTI problems, but the included \verb.clang++. compiler
  works fine.
\item Optional.  If you would like to use the parallel capabilities of
  \Spheral\ you need to install an implementation of \verb.MPI..  The MacPorts
  (\ref{free.sec}.\ref{Macports.sec}) project's distribution of
  \verb.MPICH2. works well.  Be sure to install the \verb.clang. variant if you
  use Macports as follows:
\begin{verbatim}
    sudo port install mpich2 +clang
\end{verbatim}
\item From this point you can configure and build as usual.
\end{enumerate}

\subsection{Configuration examples}
When configuring your build, bear in mind the limitation that we require a
reasonably standard compliant C++ compiler.  If you encounter problems with
compiling the C++ portions of \Spheral, often using a more up to date compiler
(such as the freely available \verb.gcc/g++. suite) will resolve such issues.

The following are some generic configuration examples:

\begin{itemize}
\item To build a serial debugable version, assuming the default compiler choices
are OK:
\begin{verbatim}
./configure --with-compilers=gnu --without-mpi
\end{verbatim}

\item To build a similar serial optimized code
\begin{verbatim}
./configure --with-compilers=gnu --without-mpi --with-opt=2 --with-dbc=none
\end{verbatim}

\item To build an optimized parallel executable, manually setting the compiler
choices:
\begin{verbatim}
./configure --with-opt=2 --with-dbc=none \
            --with-CC=gcc --with-CXX=g++ \
            --with-MPICC=mpicc --with-MPICXX=mpiCC
\end{verbatim}

\item To build optimized with the intel compiler suite:
\begin{verbatim}
./configure --with-compilers=intel --with-opt=2 --with-dbc=none
\end{verbatim}

\item You can override any of the individual compiler choices as needed:
\begin{verbatim}
./configure --with-compilers=gnu --with-opt=2 --with-dbc=none \
    --with-CC=gcc-3.4.3 --with-CXX=g++-3.4.3 \
    --with-MPICC=/usr/local/wackycompilerfrontend/mpicc \
    --with-MPICXX=/usr/local/wackycompilerfrontend/mpiCC
\end{verbatim}

\end{itemize}

%-------------------------------------------------------------------------------
\section{Building the code}

Once you have configured \Spheral, building is simply a matter of executing {\tt
make}.  \Spheral's build system is based upon Gnu's \verb+make+, which is
available on most systems.  On a Linux based system \verb+make+ is the Gnu make,
while on many other systems Gnu make is installed as \verb+gmake+.  Building the
code from scratch can be quite a long process, particularly while building all
the third party packages.  Fortunately, if you are building on a multiprocessor
SMP machine you can take advantage of Gnu make's parallel build facilities via
the \verb+-j+ option to vastly speed up the compile process.  For instance, on a
typical AIX machine at LLNL a serial build might take several hours.  However,
by building 10 way (\verb+gmake -j 10+), the full compile can be completed in
less than half an hour.

%-------------------------------------------------------------------------------
\section{C++ only builds}

It is also possible to build \Spheral\ as a C++ only target.  In this mode
\Spheral\ skips building the third party libraries and python interface, and
instead builds only the C++ source code libraries.  The libraries and headers
are installed under the \verb.prefix. selected during the configuration process.
Additionally, it is up to the user to provide where the necessary third party
libraries are installed via the \verb.EXTRAINCLUDES. and
\verb.EXTRALIBS. compile arguments.

Consider the following example.  We want to install a C++ only optimized build under
\verb./usr/local/Spheral/cxx..  We have previously installed the necessary
third party libraries (with their headers) under some place we'll refer to as
\verb.TPL_TOP..  We can configure and build as

\begin{verbatim}
./configure --with-cxxonly --prefix=/usr/local/Spheral/cxx \
            --with-opt=3 --with-dbc=none 
make cxxonly EXTRAINCLUDES='-I TPL_TOP/include \
             EXTRALIBS='-L TPL_TOP/lib'
\end{verbatim}

This will resulting \Spheral\ headers will be installed under
\verb./usr/local/Spheral/cxx/include. and the corresponding libraries in
\verb./usr/local/Spheral/cxx/lib..

%-------------------------------------------------------------------------------
\section{Running \Spheral}
\Spheral\ is just a set of extensions to the Python language, so running
\Spheral\ is simply a matter of producing a Python script appropriate for your
problem using the \Spheral\ extension objects (available via the
\verb+import Spheral+ command).  The \Spheral\ distribution includes several
example and test scripts which demonstrate setups for several standard test
cases.  The results of one such test (the Sod shock tube) can be found
at \newline \url{http://spheral.sf.net/SodExample.html}.

When you build \Spheral\ it installs all of its components in the top directory
where you installed the source, including Python.  For instance, if you
installed the source in \verb+/home/username/Spheral+, then after you finish
building \Spheral\ there will be a \verb+/home/username/Spheral/bin+ directory
containing a \verb+python+ executable.  To use \Spheral\ you simply fire up this
version of \verb+python+ and issue the command \verb+import Spheral+ and all of
the components will be available in the module \verb+Spheral+.  Exactly how you
run the executable will depend on your architecture (and whether you configured
for parallel or serial).  Here are some example invocations assuming you have a
python script \verb+example.py+ which contains the setup for your problem, and
your source was installed in {\tt /home/username/Spheral}:
\begin{itemize}
\item If you configured for a serial executable:
\begin{verbatim}
/home/username/Spheral/bin/python -i example.py
\end{verbatim}
\item For a 4 way parallel run on a typical MPICH based parallel machine (if you
configured for a parallel executable):
\begin{verbatim}
mpirun -np 4 /home/username/Spheral/bin/python -i example.py
\end{verbatim}
\item For a 4 way parallel run on a SLURM based parallel machine (again assuming
you configured for a parallel executable):
\begin{verbatim}
srun -n 4 -N 2 /home/username/Spheral/bin/python -i example.py
\end{verbatim}
\item Similarly for a 4 way parallel run on an IBM AIX parallel machine:
\begin{verbatim}
poe /home/username/Spheral/bin/python -i example.py -nodes 1 -procs 4
\end{verbatim}
\end{itemize}

\subsection{AIX notes}
\Spheral\ builds as a series of dynamically loading libraries, which requires
that the loader be able to resolve where these libraries are at load time.
Unfortunately we are currently having some problem with the rpath on AIX, so the
user must explicitly set their LIBPATH environment variable in order for the
loader to perform this resolution correctly at run time.  The
\Spheral\ libraries are installed in {\em topdir}{\tt
  /lib/python2.4/site-packages/Spheral}, where {\em topdir} is the directory
where you installed the \Spheral\ source.  On AIX set your LIBPATH environment
variable to this path when you want to run \Spheral.

%-------------------------------------------------------------------------------
\section{Testing}
\Spheral\ includes a python based automated testing system: \verb+ats+.  You
can use this to run a set of test cases included with the distribution.
\verb+ats+ is installed in the same \verb+bin+ directory where \verb+python+
and \verb+python+ are installed.  In the \verb+spheral/tests+ directory there is
a file called \verb+integration.ats+ which runs the set of automated tests we
currently support.  You can generate a simple help/usage description for {\tt
  ats} by executing \verb+ats -h+.  Here's an example of running the automated
tests for a serial build assuming you downloaded the \Spheral\ source into the
directory \verb+/home/username/Spheral+:
\begin{verbatim}
cd /home/username/Spheral/spheral/tests
/home/username/Spheral/bin/ats -f 'np < 2' \
                               -e /home/username/Spheral/bin/python integration.ats
\end{verbatim}
Note that we added a filter (\verb+-f 'np < 2'+) to remove any tests requiring
more than one processor.

If you built the parallel MPI enabled version of the code you would execute
\begin{verbatim}
cd /home/username/Spheral/spheral/tests
/home/username/Spheral/bin/ats -e /home/username/Spheral/bin/python integration.ats
\end{verbatim}
which will run all the tests, including the parallel cases.

%-------------------------------------------------------------------------------
\appendix
\section{Third party libraries}
\label{thirdParty.sec}
\subsection{Open source software used with \Spheral}
\label{free.sec}
Listed here are  the third party packages which  \Spheral\ uses.  These packages
are  automatically  downloaded and  built  by \Spheral\  as  part  of the  build
process,  so  you  must  have  an  active  network  connection  the  first  time
\Spheral\  is  built.  \Spheral\  benefits   greatly  from   the  open-source
availability of all of these packages,  without which this project would be much
more difficult and cumbersome.  Note that you must have the \curl\ program
available on your machine in order for \Spheral\ to download these packages for
you.  If \curl\ is not installed please download and install it yourself (as
described below) before beginning the \Spheral\ build process.
\begin{enumerate}
\item \curl: \url{http://curl.haxx.se/}
\curl is the tool that makes it possible for \Spheral\ to automatically download
the third party libraries listed here.  You must ensure that this program is
installed and available on your system before starting to build \Spheral.  This
will already be available (or easily installable) for common systems such as
Linux.  If needed however you can download this software directly from the
above URL and install it yourself.

\item \verb+Python+: \url{http://www.python.org} \label{python.item} \newline
\verb+Python+ is a simple yet powerful scripting language which \Spheral\ uses
as it's primary user interface.  \Spheral\ is in fact simply an extension of the
\verb+Python+ language.

\item \verb+Boost+: \url{http://www.boost.org} \label{boost.item} \newline
\verb+Boost+ is a collection of peer-reviewed open source libraries for use with
C++.  \Spheral\ makes use of a few \verb+Boost+ components, such as the
\verb+shared-ptr+ library.

\item \verb+NumPy+: \url{http://numpy.scipy.org/} \newline
NumPy (formerly Numeric Python) adds a fast, compact, multidimensional array
language facility to Python.

\item \verb+Gnuplot+ \label{gnuplot.item}
  \url{http://gnuplot-py.sourceforge.net/} \newline \Spheral\ uses the standard
  \verb+gnuplot+ plotting package through this Python package to provide simple
  $(x,y)$ plots.  This package is just the python frontend to \verb+gnuplot+ --
  it requires that the \verb+gnuplot+ package already be installed on your
  machine.

\item Gnu Scientific Library:
  \url{http://www.gnu.org/software/gsl/} \label{gsl.item} \newline A collection
  of useful scientific numerical algorithms coded in C.  This is one of the
  packages the user is required to download separately.  Download the 1.7
  release, which can be found at \newline
  \url{ftp://ftp.gnu.org/gnu/gsl/gsl-1.10.tar.gz}

\item \verb+SciPy+: \url{http://www.scipy.org/} \newline A python based
  collection of numerical algorithms.  This not actually required by \Spheral
  (and is not included in the distribution), but if you install {\tt SciPy} in
  the \verb+spheral/src/thirdPartyLibs+ directory and configure
  \Spheral\ appropriately, \Spheral\ will build and install it.

\item \verb+pybindgen+: \url{http://pybindgen.googlecode.com} \newline
  The code generator we use to bind/expose the C++ \Spheral\ code to python.

\item\verb+Macports+: \url{http://www.macports.org/} \newline
\label{Macports.sec}
A collection of standard open-source Unix software for Mac OS X.

\end{enumerate}

\subsection{Non-free software}
\label{nonfree.sec}
Here we list software which is not necessarily open-source and freely available,
but which \Spheral\ can optionally use if so desired.
\newcommand{\Parmetis}{{\tt ParMETIS}}
\begin{enumerate}
\item \Parmetis: \label{parmetis.item}
  \url{http://glaros.dtc.umn.edu/gkhome/views/metis/} \newline \Parmetis\ is a
  distributed parallel graph portioning tool which \Spheral\ relies upon for
  it's parallel distributed domain decompositioning in 2-D and 3-D.
  \Parmetis\ is used both to create an initial domain decomposition and to
  periodically redistribute the nodes in order to maintain parallel efficiency
  as the nodes move about.
\end{enumerate}

\end{document}
