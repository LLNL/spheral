The interface to FractalGravity is so simple that even a government employee can figure it out.

The user has to create a Fractal_Memory object, *PFM, before invoking FractalGravity. 
*PFM is constructed with a set of parameters that are really obvious. 
There is a large set of internal parameters for *PFM that can be set by an advanced user.
Parameters cannot be changed during the simulation. Well, that is not quite true; but don't do it.
The *PFM object is kept throughout the simulation and is only deleted after the last FractalGravity call.
All communication between the user and FractalGravity is handled via *PFM.

Before invoking FractalGravity we must construct a Fractal object, *PF, this is hidden from the user.
*PF gets all its parameters from *PFM.
The user must feed the particle data to *PF via *PFM.
The user must specify her computational cube so that FractalGravity can scale the positions down to the unit cube.
User does no scaling of positions and masses. 
If isolated BC, particles outside the cube are ignored.
If periodic BC, particles outside the cube are wrapped into the box.
The *PF object is deleted after each gravity calculation together with all tree information.

When it is all over, user must call to delete *PFM object.

If user wishes to make changes to the *PFM object, she must first delete the object and then rebuild it with new parameters.

Loadbalancing is dynamic, it is done everytime FractalGravity is invoked. All volumes are boxes.
Each node is responsible for a volume of the unit cube. If ``balance=0'' each node gets the same size volume. If ``balance=1''
the balancing algorithm equals a linear combination of the number of particles and the number of points in a volume.

First thing to do is create a Fractal_Memory object and give it the proper parameters.
This is done once.

  Fractal_Memory* FractalGravityFirstTime(int NumberParticles,
					  int balance,
					  int FractalNodes0,
					  int FractalNodes1,
					  int FractalNodes2,
					  int FFTNodes,
					  bool Periodic,
					  bool Debug,
					  int GridLength,
					  int Padding,
					  int LevelMax,
					  int MinimumNumber,
					  int MaxHypreIterations,
					  double HypreTolerance,
					  string BaseDirectory,
					  string RunIdentifier,
					  bool TimeTrial,
					  MPI_Comm& TalkToMe);

Then you can do the real calculation

  void FractalGravity(Fractal_Memory* PFM,int NumberParticles,vector <double>& xmin,vector <double>& xmax, double G)

When you are all finished you need to clean up.

  void FractalGravityFinal(Fractal_Memory* PFM)
